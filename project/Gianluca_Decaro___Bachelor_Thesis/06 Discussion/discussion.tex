% Kapitel 9 - Fazit
\section{Discussion}
%temp
This research aimed to address a critical gap in automotive \gls{HMI} for future \gls{AV}s by investigating a novel interface paradigm, and how different levels of feedforward could support intuitive and engaging interactions with a novel, non-wearable textile interface. To achieve this, a user study was conducted that evaluated the performance, usability, and user experience of three distinct feedforward conditions: a baseline with only Inherent physical cues, a condition with seamlessly integrated abstract light-based cues, and a condition with explicit text-based cues on an accompanying windshield display. This chapter will now discuss the key findings from this study, interpreting the results in relation to the primary research questions and exploring their implications for the design of future in-vehicle interfaces.

\subsection{RQ1: Observed Performance and Intuitiveness}

To address \textbf{RQ1: How does the level of feedforward affect task success rates and user errors when interacting with the textile interface without prior training?}, this section examines participants’ observed performance across the feedforward conditions.

\paragraph{Steep Learning Curve}
A primary finding from the objective performance data is the evidence of a \textbf{steep but rapid learning curve across all three feedforward conditions}. This trajectory is clearly visible in the performance metrics across the four scenarios. Success rates were lowest in the initial \hyperref[ws2-welcome-scenario]{Welcome} and \hyperref[ms2-navigate-home-menu]{Music} Scenarios, while the average number of failed attempts and the need for external hints were almost exclusively concentrated in these first two scenarios. The \hyperref[ms2-navigate-home-menu]{Music} Scenario consistently represented the point of peak difficulty for participants, with the lowest success rates and the highest number of failures and hints required. Following this initial phase, performance improved dramatically for all groups; in the final \hyperref[ss2-navigate-home-menu]{Spotlight} and \hyperref[cs2-incoming-call]{Call} Scenarios, success rates were high, with failed attempts dropping to near-zero levels and a full absent of required external hints from the study conductor.

This trend was also reflected in the qualitative feedback, where numerous participants described the steep learning curve with an initial familiarization phase marked by challenge, which they attributed to the inherent novelty of interacting with a textile-based system. This is unsurprising, given that the participant sample reported very limited prior experience with textile interfaces. The initial performance struggles can therefore be largely understood as a direct consequence of this novelty. 

\paragraph{Conflict Between Design and Intuition}
A primary cause for the poor initial performance scores was the "Marble" metaphor, first introduced in the \hyperref[ws2-destination-selection]{Destination Selection} task of the Welcome Scenario. This task revealed a significant conflict between the designed "swipe-inwards-to-select" interaction and the users' established mental models. Qualitative feedback revealed that participants' first instincts were based on their deep-seated prior knowledge from touch devices, their "knowledge in the head" \cite{norman_design_2013}, which has led to the observed failed gestures; tapping of the pleated area or swiping outwards towards the target. The interface's intended gesture, guided by its inherent physical and graphical cues through design, or "knowledge in the world" \cite{norman_design_2013}, in many cases revealed to be not strong enough to override these powerful, pre-existing habits. 

This finding suggests that the inherent cues of the pleated surface, intended as a visual and textile signifier \cite{mlakar_signifiers_2025} for swiping, were insufficient to override users' powerful expectations. 
While prior research by Jiang et al. \cite{jiang_gesfabri_2022} found that, in isolation, pleats successfully afford stroking gestures, this study indicates 
% that the effectiveness of a single physical cue can be diminished within a holistic system with competing information from a GUI and other physical elements. 
that in a holistic system with competing visual information from a \gls{GUI} and other physical elements, these affordances can be diminished or overridden by users' stronger, pre-existing mental models.
The issue was potentially compounded by a physical false affordance \cite{gaver_technology_1991} for the interaction directionality, as some participants noted that the friction of the pleats felt smoother when swiping outwards, discouraging the intended gesture. 

The augmented light cues proved to be the most effective solution to this conflict. 
By providing animated, color-coded guidance directly on the textile surface, the light cues successfully supported the users' reported primary learning strategy of visual mapping. This "visual-first" approach is consistent with findings from Mlakar et al. \cite{mlakar_signifiers_2025} and was effective because the cues successfully balanced both affordance clarification and function-revealing information. This resulted in participants within the Light Cues condition requiring significantly fewer failed attempts and no external hints during the \hyperref[ws2-welcome-scenario]{Welcome Scenario}. 
In contrast, the Text Cues were surprisingly ineffective at preventing initial errors in this scenario, resulting in a performance profile nearly identical to that of the Inherent Cues. Observations and qualitative feedback revealed that participants often did not notice the text hints until \textit{after} their initial, intuitive attempts had failed, thus negating the cue's preventative benefit.
% Interestingly, the Text Cues condition performed no better than the Inherent Cues in this initial scenario. Qualitative feedback suggests this is because participants often acted on their intuition first, overseeing the text cues and only looking to the WSD for further instructions \textit{after} their initial attempt had failed, thus negating the cue's preventative benefit.


\paragraph{The Challenge of Discoverability}
The performance dip observed in the \hyperref[ms2-navigate-home-menu]{Music Scenario} across all conditions can be attributed to several challenging tasks that exposed issues with discoverability and conceptual connection. For the \hyperref[ms2-start-playback]{Start Playback} task, the conceptual link between the on-screen vinyl metaphor and the physical dial was not immediately apparent, causing most users to default to their prior "knowledge in the head" and attempt a simple tap to play the music. More critically, both the \hyperref[ms2-adjust-volume]{Adjust Volume} and \hyperref[ms2-back-to-home-menu]{Back to Home Menu} tasks suffered from a fundamental discoverability problem. 

Qualitative feedback revealed that the primary failure of the peripheral volume slider and home loop was a "perceptual disconnect"; many participants initially mistook them for decorative or structural components rather than interactive elements. Participants attributed this to several factors: their visual focus was drawn to the dynamic, shape-shifting hub of the interface; the static elements were not represented on the \gls{GUI}; and they were perceived as a different concept from the rest of the interactive surface. The novel pull-gesture for the loop also clashed with user expectations for in-vehicle controls and created hesitation, as users were reluctant to apply force to a research prototype. It is important to consider that this \textbf{study context} may have amplified the discoverability issue. In a finished production vehicle, users might assume every element has a function. However, within a research setting, participants may have been more likely to dismiss unfamiliar elements as low-fidelity artifacts of the prototype rather than intentional, interactive controls. 

It is important to note that this discoverability issue was first identified during the formative pre-study. In response, several refinements were made to the prototype's second version specifically to strengthen the signifiers of these peripheral elements, such as increasing their visual contrast. The fact that the "perceptual disconnect" persisted as a major usability failure in the final study, despite these targeted improvements, makes the finding more robust. This suggests that the problem is not merely one of weak signifiers, but a more fundamental issue of attentional focus, where even enhanced peripheral cues can be overlooked when a user's attention is captured by a central, dynamic interaction zone. 

Crucially, the issue with these elements was one of \textbf{discoverability}, not of inherent \textbf{usability}. Once participants located the slider and the loop, their physical forms provided strong affordances for the correct interaction, confirming suggestions from prior research \cite{mlakar_signifiers_2025, mlakar_design_2020}. The data confirms this, as hints provided by the study conductor were exclusively for locating the elements; participants never required help in determining \textit{how} to perform the slide or pull gesture itself.

This distinction explains the significant performance difference between the feedforward conditions. The augmented light cues were highly effective because they solved the discoverability problem by directing the user's visual focus to the correct element, confirming the "visual-first" approach identified by Mlakar et al. \cite{mlakar_signifiers_2025}. In contrast, the Inherent and Text Cues conditions performed poorly because neither the physical design alone nor the remote text instructions were sufficient to overcome this initial perceptual barrier.

\paragraph{Successful Transfer of Prior Knowledge}
The strategy of leveraging prior knowledge from touch devices was not solely a pitfall; in some cases, it proved to be a significant asset. This was most evident in the \hyperref[ss2-navigate-home-menu]{Spotlight Scenario}. Despite introducing another unfamiliar interaction on a large, featureless \gls{2D} plane, this task resulted in very high success rates and required almost no hints across all conditions.

The success of this interaction can be attributed to its close resemblance to familiar paradigms like a laptop trackpad or a large touchscreen. The pointing task involved \textbf{direct spatial manipulation} of an on-screen element, a mental model that is deeply ingrained in users' "knowledge in the head". Unlike the "Marble" metaphor, which conflicted with established habits, the "Spotlight" interaction allowed for a direct and successful transfer of existing skills, making it feel immediately intuitive to participants.

\paragraph{Consolidating Learning and Transfer of Skills }
The final \hyperref[cs2-incoming-call]{Call Scenario} served as a strong indicator of the interface's overall learnability and the successful transfer of learned skills. By this stage of the study, performance was high across all three conditions, with participants achieving high success rates while requiring minimal hints or repeated attempts. Although the tasks were presented in a new context, participants were able to successfully recall and apply the various interaction methods that they had learned in the preceding scenarios. This demonstrates that once the initial learning curve was overcome, the system's interaction model was consistent and predictable enough for users to apply their new knowledge effectively, regardless of the feedforward condition. 
 
This successful transfer of learning is strongly supported by the qualitative feedback, where numerous participants noted that after overcoming an initial hurdle, the interface felt logical and easy to use. 
The overall development of participants performance regardless of their feedforward condition indicates that the primary challenge of this novel \gls{HMI} is not in its ongoing \textbf{use}, but in the initial \textbf{learning phase}. It is precisely within this critical learning period that the differences between the feedforward conditions become most apparent, with effective guidance significantly reducing the initial friction and improving the overall performance trajectory. 

%\paragraph{RQ1: How does the level of feedforward affect task success rates and user errors when interacting with the textile interface without prior training?}

\paragraph{Effect of Feedforward on Observed Performance and Intuitiveness}
In summary, the choice of feedforward condition had a profound impact on observed performance and intuitiveness. The \textbf{Light Cues} condition consistently and significantly outperformed the others, resulting in higher success rates and fewer failed attempts. Qualitative feedback reveals why this method was so effective: participants perceived the lights as a highly effective and supportive layer of guidance that excelled at clarifying affordances, aiding the discoverability of interactive elements, and revealing their function. This finding not only aligns with  the work of Dong \cite{dong_disappearing_2019}, who found that a combination of shape-changing physical cues and dynamic light patterns resulted in the highest perceived affordance for a textile interface, but also expands upon it. While Dong's \cite{dong_disappearing_2019} research focused on a single, isolated element, the present study demonstrates that this principle holds true even within a complex, holistic system with multiple competing interaction zones and tasks.

In contrast, there were no significant performance differences between the \textbf{Text Cues} and \textbf{Inherent Cues} conditions, both of which performed significantly worse than the Light Cues condition. The Inherent Cues suffered because its physical signifiers were often too ambiguous or weak to override users' powerful pre-existing mental models, and in some cases, key interactive elements were not discovered at all. Similarly, while participants found the text cues generally helpful, their effectiveness was undermined by several key issues. Unlike the light cues, the remote text instructions did not support the crucial user strategy of visually mapping \gls{GUI} elements onto the physical textile interface. Instead of guiding the user's hand, they required a multi-step cognitive process of reading, interpreting, and then locating the correct control. This was compounded by discoverability issues, as some users only noticed the instructions after their initial intuitive attempts had already failed.

% In contrast, there were no significant performance differences between the \textbf{Text Cues} and \textbf{Inherent Cues} conditions. While participants found the text cues "generally helpful", their effectiveness was undermined by several key issues. Unlike the light cues, the remote text instructions did not support the crucial user strategy of visually mapping GUI elements onto the physical textile interface. Instead of guiding the user's hand, they required a multi-step cognitive process of reading, interpreting, and then locating the correct control. This was compounded by discoverability issues, as some users only noticed the instructions after their initial intuitive attempts had already failed.
 
It is also worth considering the participant demographics in this context. Although all participants were fluent in English, none were native speakers. While no participant explicitly mentioned language as a barrier, it is possible that processing semantic instructions in a non-native language contributed to the cognitive load or potential for ambiguity. This consideration highlights a clear, inherent advantage of the abstract light cues, which are freed from language barriers, making the interface more universally understandable.

\subsection{RQ2: Perceived Intuitiveness and Usability}

To explore \textbf{RQ2: How do different levels of feedforward influence the perceived intuitiveness and usability of the textile interface?}, this section examines how participants evaluated the system’s usability and ease of learning.

\paragraph{Augmented Feedforward as a Prerequisite for Usability}
The critical role of augmented feedforward is strongly reflected in the perceived usability of the interface, as measured by the \gls{SUS}. A significant gap was observed between the conditions. The Inherent Cues condition received a score of $62.25$, which falls into the "OK" range and indicates usability issues. In stark contrast, both the Light Cues ($88.50$) and Text Cues ($80.50$) conditions scored in the "excellent" usability tier. This demonstrates that for a novel, holistic system, the inherent physical properties alone do not provide sufficient guidance, making some form of augmented feedforward a fundamental requirement to create a usable system for first-time users. 
% This finding adds important nuance to prior research in this area. While studies have successfully shown that isolated physical properties can provide strong and intuitive guidance, the poor usability score of the Inherent Cues condition ($M = 62.25$) suggest this principle may not scale to a more complex interface.
This finding adds important nuance to prior research that has successfully shown that isolated physical properties can provide strong and intuitive guidance \cite{dong_disappearing_2019, jiang_gesfabri_2022, mlakar_signifiers_2025, mlakar_design_2020}. In fact, the physical design of the prototype was intentionally based on these findings. However, the poor usability score of the Inherent Cues condition ($62.25$) suggests that these principles do not necessarily scale to a more complex interface. It appears that in a holistic system with multiple competing elements and a visually dominant \gls{GUI}, the guidance provided by individual physical cues is not sufficient on its own.

\paragraph{Performance vs. Perception in Augmented Cues}
Interestingly, while the objective performance data showed a clear advantage for the Light Cues, there was no statistically significant difference in the final \gls{SUS} scores between the two augmented conditions. This suggests that despite the initial friction and higher error rates experienced by the Text Cues group, both augmented methods ultimately provided a sufficient "feeling of support", as mentioned in qualitative feedback, for participants to perceive the final product as highly usable. This highlights a key distinction between objective performance during a learning phase and a user's summative and subjective assessment of overall usability. It is worth noting, however, that the mean score for the Light Cues condition was still eight points higher than for the Text Cues condition, suggesting a trend in its favor, in line with the previously discussed observations.

\paragraph{Bridging the Learnability Gap}
While the \gls{SUS} provides a broad measure of usability, the \gls{QUESI} results offer a deeper insight into the specific components of perceived intuitiveness. Here, the perception of the interface as "intuitive" was heavily dependent on the presence of augmented feedforward, as revealed by the \gls{QUESI} results. Both the Light Cues and Text Cues conditions were rated as requiring significantly less Effort to Learn and feeling significantly more Familiar than the Inherent Cues condition alone.

This directly addresses the concept of intuitive use, which is characterized by a low learning effort and a sense of ease that comes from leveraging a user's existing knowledge \cite{blackler_investigating_2010, khan_intuitive_2017, mignonneau_designing_2005}. The findings demonstrate that for this novel \gls{HMI}, intuitiveness was not an inherent property of the physical form but was achieved through effective guidance. The augmented cues successfully bridged what Schäfer et al. \cite{schafer_whats_2023} refer to as the "learnability gap" for textile interfaces by providing clear "knowledge in the world" to compensate for the users' lack of "knowledge in the head" \cite{norman_design_2013, schafer_whats_2023}. This external guidance is what made the new system feel familiar and easy to grasp, fulfilling a key requirement for intuitive interaction.

\paragraph{The "Psychological Safety Net" of Textual Instructions}
An interesting finding from the \gls{QUESI} results is that while Text Cues did not significantly improve initial objective performance compared to the Inherent Cues, they had a significant positive impact on the perception of intuitiveness. This can be understood through the concept of a "psychological safety net". For the Inherent Cues group, the learning process was one of pure trial-and-error. In contrast, even when participants in the Text Cues group made initial mistakes, they knew a source of truth was available to fall back on. This safety net likely reduced the anxiety associated with the learning process, making it feel less effortful and more guided.

Furthermore, the method of guidance itself, of reading instructions on a screen, is a highly familiar way to learn a new product, as expressed by some in the interviews. It is plausible that this familiarity with the learning method transferred to the novel interface, making the product itself feel less intimidating and more familiar overall. These psychological factors help explain why the Text Cues condition was rated as requiring significantly less Effort to Learn and feeling significantly more Familiar than the Inherent Cues condition, despite their similar initial performance profiles.


% A particularly nuanced insight from the QUESI results is the clear distinction between the initial learning process and the ongoing use of the interface. While the feedforward conditions had a significant impact on the perceived Effort to Learn, there were no statistically significant differences in the ratings for Low Subjective Mental Workload, High Perceived Achievement of Goals, or Low Perceived Error Rate.
\paragraph{The Hurdle of Learning vs. Using}
The subjective \gls{QUESI} results provide a compelling explanation for the steep but rapid learning curve observed in the objective performance data. While there was a significant difference between conditions in the perceived Effort to Learn, no such differences were found for Low Subjective Mental Workload, High Perceived Achievement of Goals, or Low Perceived Error Rate; in fact, all groups reported high, positive scores on these three scales.
This suggests that the primary challenge of this novel \gls{HMI} lies in the initial learning phase, not in its sustained use. Once participants, including those in the Inherent Cues condition, overcame the initial hurdle, they felt just as competent and successful as those who received augmented guidance. This indicates that the core interaction model, once understood, is not inherently difficult or mentally taxing to operate, reinforcing that the main usability barrier is one of initial guidance rather than fundamental design.

\subsection{RQ3: Perceived User Experience}

To answer \textbf{RQ3: What impact do the different feedforward levels have on the hedonic and pragmatic qualities of the user experience?}, this section looks at how users experienced the interface as a whole, beyond just task success.

\paragraph{The \gls{UX} Journey: Mirroring the Learning Curve}
The trajectory of the user experience, as captured by the \gls{UX} Curve, shows a strong correlation with the objective performance data, reinforcing the link between the learning curve and perceived \gls{UX}. This is clearly demonstrated by the corresponding dip in both success rates and \gls{UX} ratings across all conditions during the challenging Music Scenario. The journey of the Inherent Cues group provides a stark illustration of this connection: their significant initial performance struggles led directly to a worse \gls{UX} rating, a sentiment echoed in the qualitative feedback where these participants expressed frustration and a desire for an initial onboarding. In contrast, the consistently high performance of the Light Cues group was mirrored by their stable, highly positive \gls{UX} ratings throughout the study.

A notable finding is that all conditions converged on an almost equally high \gls{UX} rating in the final scenario, suggesting the long-term potential of the tactile interface is positive regardless of the initial guidance. However, the key difference lies in the journey to competence. The augmented light cues were critical in making this journey smoother and less frustrating, which is vital for user acceptance of this \gls{HMI} paradigm. This is supported by qualitative findings where elements that caused initial frustration due to poor discoverability, like the volume slider and home loop, were later praised by participants for their pleasant tactile feel and fun or playful nature once they were learned. This indicates that the core design is hedonically strong, and the primary role of the light cues was to make this enjoyable experience accessible from the start.

\paragraph{Tolerance for New Experiences}
It is important to note that despite the objective usability problems, the subjective user experience remained largely positive, even during the steepest part of the learning curve. With the exception of the neutral ratings for the Inherent Cues group in the first two scenarios, the mean \gls{UX} Curve scores for all conditions remained high, generally above a value of 2 (on a scale where 3 is "very good").

This suggests that participants exhibited a high tolerance for the initial learning challenges of such a novel interface. This interpretation is supported by the qualitative feedback, where participants often described the learning curve as manageable and appropriate for a new product, and the process of discovery itself as "fun" and "exciting". This willingness to overcome initial hurdles could be influenced by the study's participant sample, which consisted of a young demographic that self-reported as being highly tech-savvy and very interested in new technologies. It is plausible that such a group is more open-minded and patient when learning an unfamiliar interface, particularly one they perceive as innovative and futuristic.

\paragraph{The Pragmatic Failure of Unguided Interaction}
The findings from the \gls{UEQ} further illuminate these differences, particularly regarding the interface's task-oriented qualities. The Pragmatic Quality scores, which measure the goal-oriented aspects of the experience, closely mirrored the overall usability ratings from the \gls{SUS}. Both the Light Cues ($M = 2.06$) and Text Cues ($M = 1.80$) conditions were rated as highly pragmatic, while the Inherent Cues condition scored significantly lower ($M = 0.89$) than the Light Cues.

An analysis of the subscales, which inform the pragmatic qualities, reveals the specific nature of this pragmatic failure. The low score for the Inherent Cues condition was primarily driven by very low ratings for Perspicuity (clarity and learnability) and Dependability (predictability and control). This demonstrates that the lack of augmented guidance did not just make the interface harder to use in a general sense; it specifically made the experience feel unclear, unpredictable, and less controllable for the user.

\paragraph{The Inherent Hedonic Appeal of the Tactile Interface}
The \gls{UEQ} results also reveal a complex relationship between the interface's pragmatic (task-oriented) and hedonic (non-task-oriented) qualities, a distinction that is central to the changing nature of the in-vehicle experience in automated driving \cite{detjen_how_2021}. The data supports the idea that pragmatic failures can directly harm the hedonic experience; the frustration and confusion from the poor usability in the Inherent Cues condition demonstrably lowered its scores for Attractiveness and overall Hedonic Quality compared to the guided conditions. This shows how interconnected these qualities are, as it is difficult for a product to be perceived as "fun" if it is not first functional.

However, a key finding is that despite these usability challenges, the hedonic ratings for the Inherent Cues condition remained highly positive ($M = 1.95$). This is because the core interface is inherently stimulating and novel. Across all three conditions, participants gave very high ratings for Novelty and Stimulation, an assessment strongly supported by the qualitative feedback where the interface was consistently described as "innovative", "fun", and "exciting". Specifically, participants highlighted the "Rotary Dial" as a standout success, praising its playful gesture and engaging interaction metaphors like the spinning vinyl record and rotary phone. This can be seen as a practical application of possibility-driven design \cite{desmet_towards_2012}, which successfully transformed a simple control into a fun and positive experience for the user. The high Stimulation ratings can also be attributed to the pleasant sensory experience of the textile surface itself, which users noted provided enjoyable tactile feedback that has become rare in today's digital devices.
This indicates that the problem with the Inherent Cues condition wasn't that the interface was unenjoyable, but that the lack of guidance created a usability barrier that prevented users from fully appreciating its positive qualities. This reframes the research challenge as not just finding a usable interface, but finding the best way to unlock the potential of an already stimulating one. 

This strong inherent appeal may also explain why the overall user experience, as seen in the \gls{UX} Curve, did not become more negative for the Inherent Cues group during the difficult initial scenarios. It is plausible that the high hedonic quality of the novel interaction acted as a "frustration buffer", providing a positive sensory and exploratory experience that helped users tolerate and overcome the initial performance issues.

\paragraph{The Impact of Feedforward on User Experience}
In summary, the choice of feedforward had a clear impact on the overall perceived user experience. The results consistently show that both augmented feedforward conditions created a superior experience compared to relying on inherent cues alone. This was reflected across the main \gls{UEQ} scales, where both Light Cues and Text Cues led to significantly higher ratings for Hedonic Quality and showed strong trends towards higher ratings for Pragmatic Quality and Attractiveness.

When comparing the two augmented methods, while no statistically significant differences were found, a consistent trend emerged in the descriptive data. The Light Cues condition was generally perceived as the most attractive and pragmatically sound, suggesting it may offer the most well-rounded and positive user experience overall. This conclusion aligns with the findings of Dong \cite{dong_disappearing_2019}, who also determined that a combination of shape-changing cues and dynamic light patterns resulted in the best user experience in their study of a textile interface.

\subsection{RQ4: Emotional Response}

To address the final research question, \textbf{RQ4: How do the different feedforward levels affect the user’s emotional response during the interaction?} this section investigates how each type of feedforward shaped participants’ emotional reactions throughout the study.

\paragraph{The Emotional Cost of Poor Usability}
The emotional responses measured by the \gls{meCUE} questionnaire reinforce the findings from both the objective performance and subjective usability data, revealing a direct emotional cost associated with poor usability. The Inherent Cues condition, which had the most performance issues and the lowest usability ratings, also generated a significantly less positive emotional profile. This group reported significantly fewer Positive Emotions and, conversely, significantly more Negative Emotions than the augmented feedforward conditions. This strong correlation suggests that the objective struggles, such as the failed attempts, the need for hints, and the general confusion, translated directly into a negative emotional experience for the user, likely in the form of frustration.

\paragraph{Augmented Cues as an Emotional Buffer}

In contrast, both the Light Cues and Text Cues conditions acted as an "emotional buffer", successfully preventing the frustration seen in the baseline condition. Both augmented methods significantly reduced negative emotions and fostered a more positive emotional experience overall. This positive emotional outcome can be directly linked to the high pragmatic and hedonic qualities reported for these conditions. By making the interface feel clear, dependable, and easy to learn (pragmatic qualities), the augmented cues removed the primary sources of frustration. Simultaneously, by making the novel and stimulating aspects of the interface more accessible (hedonic qualities), they allowed participants to fully engage with the enjoyable aspects of the experience. Thus, directly fulfilling the need to create the kind of "Positive Experiences" that are considered critical for fostering user acceptance and trust in future \gls{AV}s \cite{detjen_how_2021}.

% These emotional outcomes are not of little importance; they have significant implications for the broader goal of fostering user acceptance of AVs. As established in the literature review, creating "Positive Experiences" is a crucial design challenge to combat anticipated boredom and build the necessary trust for widespread adoption of AV technology. The findings from this study provide empirical evidence on how to achieve this. By successfully preventing frustration and making the novel aspects of the interface accessible and enjoyable, the augmented feedforward conditions fulfilled a core psychological need for stimulation and positive engagement. This demonstrates that the choice of feedforward is not merely a usability concern, but a critical factor in designing the kind of hedonically rich and emotionally positive experiences that will be essential for the success of future autonomous vehicles.


\subsection{Acceptance and Perceived Value of the novel HMI Paradigm}

\paragraph{Pragmatic Advantages over Conventional Interfaces}
While the primary focus of this study was on evaluating the influence of feedforward, the qualitative data revealed strong and consistent perceptions about the value of the developed \gls{HMI} paradigm itself, particularly when compared to conventional in-vehicle touchscreens. An overwhelming majority of participants articulated clear advantages that point to a high potential for user acceptance. The most frequently cited benefits were superior ergonomics and the potential for safer, eyes-free interaction with the textile interface as input modality. Participants contrasted the relaxed posture of using the armrest-based controller with the physical strain of reaching for a dashboard-mounted touchscreen, which they associated with "gorilla arm" fatigue \cite{hansberger_dispelling_2017, sathyan_study_2020}. The textile interface, with its rich tactile cues, was seen as a solution to the high visual demand of touchscreens, with a vast majority expressing strong confidence in their ability to eventually operate it without looking. Participants identified this as a key advantage, suggesting it could help mitigate the motion sickness often caused by looking down at screens in a moving vehicle \cite{diels_self-driving_2016}, while also allowing them to observe the vehicle's automated driving behavior to build trust in the system.

\paragraph{Hedonic Richness and Versatility}
Beyond these pragmatic benefits, participants perceived the overall user experience as more hedonically rich and engaging. The interface was described as feeling more "calming", "luxurious", and "integrated" into the vehicle interior than a standard screen. Users found the haptic nature of the interaction to be more "exciting" and "pleasant", and even noted practical benefits such as the avoidance of fingerprints. 
% Furthermore, the dynamic nature of the interface, which allows the interactive surface to "hide" and become completely flat, was seen as having unique potential to support other NDRAs, with one participant noting its suitability as a "mouse pad" for working in the vehicle with mobile devices \cite{ahram_non-driving_2020, pfleging_investigating_2016, large_design_2017}.
Furthermore, its versatility was seen as a key advantage for supporting future \gls{NDRA}s. One participant noted that its ability to become completely flat made it suitable for other activities like working with a computer mouse on it \cite{ahram_non-driving_2020, large_design_2017, pfleging_investigating_2016}, and another even extrapolated that its inherent flexibility would allow it to move with the seat in a future reconfigurable interior, a possibility not afforded by fixed screens.

\paragraph{The Learning Curve as a Hurdle to Acceptance}
However, the qualitative feedback also highlighted the primary hurdle to acceptance: the steep learning curve required to overcome the ingrained habits of touchscreen use. Participants acknowledged that because they are "more used to touchscreens", the novel paradigm requires a conscious learning effort. Several participants expressed concern that while this learning curve was manageable for them, it might prove to be a significant barrier for less tech-savvy users, potentially hindering the paradigm's wider success. This suggests that while the long-term potential of the \gls{HMI} paradigm is perceived as highly valuable, its successful adoption hinges on overcoming this initial learnability challenge, reinforcing the critical importance of effective feedforward guidance.

\subsection{Design Recommendations and Implications}
The findings from this study lead to several key design recommendations. While derived from an automotive context, these principles can be generalized to the design of other novel, holistic textile interfaces that are used in conjunction with a graphical user interface.

\paragraph{Test Within a Holistic System}
Designers should not draw confident conclusions from findings on isolated interactive elements. This study found that physical cues that proved effective in prior research were insufficient within a complex system with competing visual information from a \gls{GUI} and other physical elements. Therefore, to accurately assess their effectiveness, textile interfaces should always be evaluated within their final, holistic implementation, even if individual components have been tested beforehand.

\paragraph{Provide Robust Guidance to Overcome the Learning Curve}
Due to the inherent novelty of textile interfaces, designers must anticipate a steep initial learning curve for first-time users. The findings clearly show that without sufficient guidance, users struggle, leading to poor performance and a negative emotional experience. It is therefore critical to provide robust feedforward cues that help users build a mental model quickly and ensure a positive initial journey, as this is vital for user acceptance.

\paragraph{Leverage Hedonic Qualities as a Motivator}
Designers should prioritize the hedonic qualities of textile interfaces. This study found that the interface's inherent novelty and the inclusion of playful interactions created a strong, positive hedonic experience that acted as a "frustration buffer", helping users tolerate initial usability challenges. By focusing on creating a fun and sensorially rich experience, designers can motivate users to overcome the initial learning curve associated with a novel interaction paradigm.

\paragraph{Prioritize Integrated, Non-Verbal Guidance}
While both augmented feedforward methods proved superior to inherent cues alone, the study consistently showed that integrated, non-verbal light cues outperformed external, text-based instructions. Light cues were quicker to perceive, free from language barriers, and supported the crucial user strategy of visually mapping the \gls{GUI} to the physical controller. To minimize cognitive load and the split-attention effect, designers should prioritize this type of direct, on-surface guidance. Text cues still present a viable alternative, particularly to save cost in the design process, or they could serve as a secondary "fall-back" mechanism, for example, by appearing only after a user has made an error.

\paragraph{Design for Discoverability and Anticipate User Habits}
The failure of the peripheral slider and loop was one of discoverability, not inherent usability. Designers of seamless, "calm" interfaces must use strong signifiers to prevent a "perceptual disconnect" where interactive elements are mistaken for decoration. Furthermore, designers must assume users will default to ingrained habits from touchscreens  and provide exceptionally clear guidance when introducing novel interactions that contradict these powerful, pre-existing mental models.

\paragraph{Emphasize the Paradigm's Unique Strengths}
The qualitative feedback showed that users perceive immense value in this \gls{HMI} paradigm's potential for superior ergonomics and safer, eyes-free interaction within the automotive context. Designers should focus on delivering these key advantages. This means creating interfaces with strong tactile differentiation that can be confidently operated without looking, thereby fulfilling the potential to mitigate issues like motion sickness and allowing users to build trust in the autonomous system by keeping their eyes on the road.
 
\paragraph{Using the Feedforward Matrix as a Design Tool}
To synthesize these recommendations, it is useful to return to the Feedforward Matrix proposed in the theoretical background (see Section \ref{sec:feedforward-matrix}). This framework helps to visualize the trade-offs between different types of guidance. The results of this study can be plotted onto the matrix, with each condition occupying a distinct space: the Inherent Cues condition represents the extreme of Embedded and Affordance-Clarifying guidance, while the Text Cues condition sits towards the opposite extreme of Augmented/External and Function-Revealing instruction.

The Light Cues condition, which consistently produced the best performance and user experience, occupies a balanced, hybrid position near the center, which this study identified as the "sweet spot" (see Fig. \ref{fig:sweet-spot}) . Its success comes from its ability to enhance the physical, Embedded properties of the interface with a seamlessly Augmented layer, while simultaneously balancing Affordance-Clarifying guidance (e.g., animated gestures) with Function-Revealing information (e.g., color mapping).

\begin{figure}[h!]
    \centering
    \includegraphics[width=1\linewidth]{images/Feedforward Matrix/Feedforward Matrix Sweet-Spot.png}
    \caption{The Feedforward Matrix, plotting the three experimental conditions: (a) Inherent Cues, (b) Light Cues, and (c) Text Cues. The findings indicate that the optimal experience (the "sweet spot") was achieved by the Light Cues condition, which enhanced the Embedded, Affordance-Clarifying nature of the Inherent Cues with a balanced and integrated layer of augmentation.}
    \label{fig:sweet-spot}
\end{figure}

Therefore, the Feedforward Matrix can serve as more than just an analytical tool; it can be a generative tool for designers. By consciously considering where their interventions lie on the axes of Integration and Information Type, designers can work towards creating this optimal balance, ensuring novel interfaces are not only functional but also intuitively understood and engaging from the very first use.









