%----------Zusammenfassung Englisch/Abstract----------------------------------------------------------------
\addsec{Kurzfassung}

\sloppy
Mit der Transformation des Fahrzeuginnenraums zu einem multifunktionalen Lebensraum durch automatisiertes Fahren steigt der Bedarf an neuen Mensch-Maschine-Schnittstellen. Diese müssen eine Vielzahl von fahrfremden Tätigkeiten unterstützen und dabei unaufdringlich und hedonisch ansprechend sein. Aktuelle, von Touchscreens dominierte Interaktionskonzepte weisen ergonomische Nachteile und eine hohe visuelle Beanspruchung auf. Zudem mangelt es ihnen an der für ein komfortables Ambiente erforderlichen Ästhetik. Interaktive Textilien in Kombination mit Windschutzscheiben-Displays stellen eine vielversprechende Alternative dar, da sie eine nahtlose, taktile Interaktion ermöglichen, die nur geringe Aufmerksamkeit erfordert. Ihre Neuartigkeit stellt jedoch für Erstanwender eine erhebliche Herausforderung in Bezug auf die Erlernbarkeit dar. Diese Arbeit schließt diese Lücke, indem sie untersucht, wie verschiedene Stufen von Feedforward die intuitive Interaktion mit einer textilen Schnittstelle für fahrfremde Tätigkeiten in automatisierten Fahrzeugen unterstützen können.
Im Rahmen eines Human-Centered-Design-Prozesses wurde ein High-Fidelity-Prototyp mit einer formverändernden textilen Oberfläche auf der Mittelarmlehne und einem zugehörigen Windschutzscheiben-Interface entwickelt. Dieser wurde durch eine qualitative Vorstudie $(N=5)$ iterativ verfeinert. Eine nachfolgende summative Nutzerstudie mit $N=30$ Teilnehmenden evaluierte in einem Between-Subjects-Design und mittels der Wizard-of-Oz-Methode drei Feedforward-Bedingungen: (1) Inhärente Hinweise (Cues) durch die physische Form des Interfaces, (2) abstrakte, augmentierte Licht-Cues, die direkt auf die textile Oberfläche projiziert wurden, und (3) explizite, augmentierte Text-Cues auf dem Windschutzscheiben-Display.
Die Ergebnisse zeigten, dass das Kerninterface zwar hohe inhärente hedonische Qualitäten besitzt, augmentiertes Feedforward jedoch entscheidend für eine gebrauchstaugliche und intuitive User Experience ist. In der objektiven Performanz führten die nahtlos integrierten Licht-Cues zu signifikant höheren Erfolgsraten und weniger Fehlern im Vergleich zu den Text- und inhärenten Cues. In den subjektiven Bewertungen von Usability, Intuitivität, User Experience und emotionaler Reaktion schnitten beide augmentierten Bedingungen ähnlich gut ab und wurden oft signifikant besser bewertet als die rein inhärenten Cues, ohne dass ein statistisch signifikanter Unterschied zwischen ihnen bestand. Dennoch wurde über alle Messgrößen hinweg eine durchgängige Tendenz zugunsten der Licht-Cues gegenüber den Text-Cues beobachtet.
Die Ergebnisse implizieren, dass Designer bei neuartigen textilen Systemen eine robuste Anleitung bereitstellen müssen. Diese sollte eine Balance zwischen handlungsanleitenden (affordance-clarifying) und funktionserklärenden (function-revealing) Hinweisen schaffen, um die Erlernbarkeitslücke zu schließen und Nutzern zu helfen, eine steile anfängliche Lernkurve zu überwinden. Integrierte, nonverbale Licht-Cues stellen dabei einen optimalen Kompromiss („Sweet Spot“) dar, da sie sich in der objektiven Performanz als am effektivsten erwiesen und gleichzeitig einen durchgängigen subjektiven Vorteil zeigten. Die Arbeit kommt zu dem Schluss, dass das vorgeschlagene \gls{HMI}-Paradigma aufgrund seiner wahrgenommenen ergonomischen und hedonischen Vorteile gegenüber konventionellen Bildschirmen ein hohes Potenzial für die Nutzerakzeptanz besitzt. Sein Erfolg hängt jedoch von der Implementierung eines robusten, integrierten Feedforwards ab, um eine positive und intuitive erste Nutzererfahrung zu gewährleisten. Zukünftige Forschung sollte diese Ergebnisse in realistischeren, dynamischen Umgebungen und mit diverseren Nutzergruppen validieren.