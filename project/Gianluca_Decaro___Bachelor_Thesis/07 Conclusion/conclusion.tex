% Kapitel 9 - Fazit

%\input{chap9x} %chap9_futurework_limitations}
\section{Conclusion}
Following an adapted "Quad-Diamond" design process, this work began with an extensive review of the theoretical background to define the challenges and opportunities for \gls{HMI} design in future autonomous vehicles. Based on these insights, a novel paradigm combining a textile interface with a windshield display was conceptualized and developed into a high-fidelity prototype. This prototype was then iteratively refined through a qualitative pre-study before being evaluated in a final, summative user study that compared the effectiveness of three distinct feedforward levels: Inherent Cues, Abstract Light Cues, and Explicit Text Cues.

% The study revealed that while the core textile interface is inherently stimulating and novel, its usability and the overall user experience are critically dependent on augmented feedforward. The results consistently showed that integrated, abstract light cues significantly outperformed both inherent physical cues alone and remote, text-based instructions, leading to better performance, higher perceived usability, and a more positive emotional experience.
The study revealed that while the core textile interface is inherently stimulating and novel, its usability and the overall user experience for first-time users is critically dependent on augmented feedforward. The results consistently demonstrated that relying on inherent physical cues alone was insufficient, leading to poor initial performance, significant usability issues, and a frustrating user experience. 
% In contrast, both augmented feedforward methods improved the experience, but the integrated, abstract light cues proved to be the superior solution. They significantly outperformed remote, text-based instructions by successfully guiding users through the steep learning curve, aiding in the discoverability of interactive elements, and ultimately creating a more intuitive, usable, and emotionally positive experience. 
In contrast, both augmented feedforward methods improved the experience, though their effectiveness varied across different measures. In terms of objective performance, the integrated, abstract light cues proved to be the superior solution, leading to significantly higher task success rates and fewer errors than the remote, text-based instructions. For subjective ratings of usability, intuitiveness, and emotional response, both augmented conditions performed closely and were rated significantly better than inherent cues alone, with no significant statistical difference found between them. Despite this, a consistent tendency favoring the light cues over text cues was observed across all subjective measures, suggesting they created the most well-rounded and positive experience by successfully guiding users through the steep learning curve and aiding in the discoverability of interactive elements.


\subsection{Limitations}

While this study provides valuable insights into the design of textile interfaces, it is important to acknowledge several limitations that may affect the generalizability of the findings.

First, the study was conducted using a \textbf{Wizard of Oz methodology}, where the system's responses were triggered by the experimenter rather than by integrated sensors. While this approach was chosen to ensure the validity of the interaction design itself, free from technological imperfections, a fully functional prototype might introduce different haptic or temporal characteristics that could influence the user experience.

Second, the evaluation took place in a \textbf{lab environment} that, while designed to simulate the front-seat experience, did not fully replicate the physical interior of a real vehicle. More importantly, the setup was static, meaning consequential factors such as road vibrations, changing light conditions, and the potential for motion sickness could not be replicated. As discussed in the theoretical background, these real-world conditions can significantly impact interaction accuracy and task appropriateness. Furthermore, the context of a research prototype may have influenced behavior, as participants were potentially more hesitant to interact forcefully or more likely to dismiss unfamiliar elements as artifacts compared to how they might behave in a finished production vehicle.

Additionally, the \textbf{participant sample} had specific characteristics that may limit the broader applicability of the results. The study involved a relatively small sample size, which may reduce the statistical power of the findings and limit their generalizability. Furthermore, participants were predominantly young, university-educated individuals, with a mean age of 24.9 years, who self-reported as being highly tech-savvy and interested in new technologies. This group may have a higher tolerance for learning novel interfaces than the general population. Additionally, as none of the participants were native English speakers, the cognitive load associated with the Text Cues condition may have been subtly amplified, potentially influencing its performance relative to the language-independent light cues.

Finally, the results may have been influenced by several well-known response biases. The \textbf{Hawthorne effect} \cite{macefield_usability_2007}, where individuals alter their behavior because they know they are being observed, was a potential factor. Participants were constantly observed by the study conductor, a necessity for the Wizard of Oz methodology, and were also aware that their actions were being video recorded. This awareness may have led them to be more focused or persistent than they would be in a naturalistic setting.  Furthermore, a \textbf{courtesy bias} \cite{jones_courtesy_1993} may have been present; although it was emphasized in the briefing that both positive and negative feedback was welcome, participants may have still hesitated to express their full criticism. This is especially relevant for the direct communication with the study conductor during the verbally assessed \gls{UX} Curve ratings and the semi-structured interviews, where the desire to be polite could have tempered negative feedback.  Lastly, the summative nature of the final questionnaires could be subject to \textbf{recency effects} \cite{natesan_cognitive_2016}, where participants’ overall evaluations may have been more heavily weighted by their positive experience end of the study, rather than the more challenging tasks in the initial scenarios.

\subsection{Future Work}

The findings and limitations of this thesis open up several promising avenues for future research. A primary next step should be to address the limitations of the present study by evaluating the \gls{HMI} paradigm with a \textbf{larger and more diverse participant sample}, particularly including older and less tech-savvy users, to see how the steep learning curve impacts different demographics. Furthermore, future studies should take place in more \textbf{realistic environments}, such as a high-fidelity driving simulator or an actual vehicle, to assess the impact of real-world factors like road vibrations and motion on interaction accuracy and user experience. The \textbf{effect of language} on the Text Cues could also be investigated by testing with native speakers of the language used in the interface.

Additionally, a longitudinal study is needed to \textbf{evaluate the long-term use} of the interface. The present study demonstrated a steep but rapid learning curve; future research should investigate how the user experience and the need for feedforward evolve once users have transitioned from novices to experts. Such a study could determine whether the augmented cues are still necessary after a period of familiarization or if the inherent physical cues alone become sufficient for efficient, eyes-free interaction once muscle memory is established.

Building on the findings of this work, future research could \textbf{explore the broader applicability and flexibility} of the interactive textile paradigm. While this study successfully evaluated a specific implementation on the central armrest for controlling a distant Windshield Display, the inherent flexibility of e-textiles allows for their integration into numerous other interior surfaces. Future work could investigate embedding these tactile controls into door panels to create personal interaction zones, or onto the rear bench to facilitate shared entertainment controls for backseat passengers. Such studies would further validate the paradigm's potential to create a truly ubiquitous and adaptable in-vehicle interactive environment.

While this study focused on an automotive context, the findings and design recommendations have the potential to be applicable to \textbf{other domains} where seamless, tactile interfaces are desirable, such as smart furniture, accessibility tools, or public installations. Future work could validate and adapt these principles by exploring their application in these different contexts.

Finally, this study explored three distinct points on the Feedforward Matrix, identifying a promising "sweet spot". Future research should conduct a \textbf{deeper exploration of the nuanced feedforward variations} within this matrix. A particularly interesting avenue, suggested by several participants in the debriefing interviews, would be to investigate on-surface augmented cues that are more explicit than abstract light but more integrated than remote text. This could include projecting symbolic icons or dynamic text labels directly onto the textile surface to test their effectiveness and placement on the matrix.