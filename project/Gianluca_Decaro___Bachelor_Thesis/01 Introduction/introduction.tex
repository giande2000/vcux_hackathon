% Chapter 1 - Introduction

\section{Introduction}
\begin{comment}
    Short Form
        1. AVs are expected to become reality in the near feature. They offer numrous benefits, one of them being the entierly new ways for passengers to use time and space in the car.
        2. The driver's role is expected to shift towards that of a passenger, enabling opportunities for new NDRTs
        3. The interior space of cars is set to change towards a multifunctional livingspace to support a multitude of NDRTs
        4. However in-vehicle interaction remain driver focused today
        5. Touchscreens have become the dominant form of interaction, replacing physical dashboard controls.
        6. From a driver's perspective, touchscreens require higher visual attention due to missing tactile cues, resulting in higher cognitive workload and distraction which is a saftey concern.
        7. From a passenger's perspective current in-vehicle interfaces are not optimized for them, maing them feel left out.
        8. While touchscreen based interfaces are no longer a saftey concern in AVs, ergonmomic and usability disatvanages still persist
        9. Current car interiors are missalined with the vision of multifunctional livingspaces, materials and user experience factors need to be reconcidered
\end{comment}
\subsection{Motivation}
In the near future, highly automated driving (\gls{HAD}) is expected to transition from prototypical cars to production vehicles \cite{pfleging_investigating_2016}, making driverless cars a plausible reality \cite{lipson_driverless_2017, litman_autonomous_2024}. 
Automated vehicles (\gls{AV}s) promise numerous benefits, including reduced traffic congestion and pollution, increased safety, and greater mobility for diverse user groups \cite{litman_autonomous_2024}, while also enabling entirely new ways to use time and space inside the car \cite{detjen_how_2021}. 


As automation levels increase, the driver’s role shifts towards that of a passenger, fundamentally changing the focus of in-vehicle interaction design \cite{berger_tactile_2019, detjen_how_2021, kun_shifting_2016, pfleging_investigating_2016}. Rather than prioritizing driver-centric interfaces aimed at minimizing distraction, automated vehicles enable users to direct their full attention to other activities, making activity-centered passenger experiences a key design consideration \cite{detjen_how_2021, kun_shifting_2016, pfleging_investigating_2016}. Consequently, future vehicle interiors must support a broader range of non-driving-related activities (\gls{NDRA}s) \cite{pfleging_non-_2015}, such as work, entertainment, socializing, and relaxation, to accommodate evolving user needs \cite{detjen_how_2021, ahram_non-driving_2020, ahram_what_2020, large_design_2017, pfleging_investigating_2016, wilson_non-driving_2022}.

With new opportunities for \gls{NDRA}s, the way users interact with the interior space of vehicles is set to change. \gls{AV}s may enable passengers to engage in relaxing activities during their commute or use travel time productively, making comfort- and entertainment-oriented needs increasingly relevant \cite{detjen_how_2021, schartmuller_automated_2020}. This shift has led to visions of the car interior evolving beyond a mere means of transport into a multifunctional living space; one that supports work, socializing, mental recreation, and play \cite{detjen_how_2021, khorsandi_fabricar_2023, schartmuller_automated_2020}. As automation and electrification advance, future vehicles will offer more flexible and adaptable spaces \cite{wilson_non-driving_2022}, transforming car interiors into environments akin to mobile offices or even tiny houses \cite{desjardins_living_2016, stevens_using_2019}, where all materials and objects seamlessly support ubiquitous \gls{NDRA}s \cite{khorsandi_fabricar_2023}. These intelligent surfaces could dynamically adapt to passengers’ needs, enabling frictionless transitions between activities.

%As automation and electrification advance, future vehicles will offer more flexible and adaptable spaces \cite{wilson_non-driving_2022}, transforming car interiors into environments akin to mobile offices or even tiny houses, where materials and objects could become interactive, seamlessly adapting to passengers' needs through ubiquitous NDRT interactions \cite{khorsandi_fabricar_2023}.
%where materials and objects seamlessly interact with passengers’ needs 
%Here, ubiquitous NDRTs will play a crucial role, allowing passengers to seamlessly transition between activities without friction, supported by interactive surfaces and intelligent materials that adapt dynamically to their needs \cite{khorsandi_fabricar_2023}.

\subsection{Problem Statement}

Despite these forward-looking visions, current vehicle interiors remain largely driver-focused. Many in-vehicle controls, such as those integrated into the steering wheel or dashboard, are primarily designed for the driver’s use \cite{khorsandi_fabricar_2023}, contrasting with passengers' strong desire for shared control over in-vehicle systems \cite{berger_designing_2021}.

Touchscreens, which are rapidly replacing traditional dashboard controls, have become the dominant form of interaction and are being increasingly integrated into vehicles by manufacturers \cite{dong_disappearing_2019, khorsandi_fabricar_2023, nanjappan_towards_2019}. For instance, Tesla's infamous interior design concept, known to diverge from design heuristics and lead to usability problems \cite{parkhurst_heuristic_2019}, can be found across their entire lineup of cars, including the Model Y \cite{noauthor_new_nodate}, world’s best-selling vehicle in 2023 \cite{noauthor_tesla_2024}. This model features a large, single, centralized touchscreen that has replaced all traditional physical in-vehicle dashboard controls. This trend is already underway and expected to continue, with more manufacturers adopting touchscreen-based controls for a large majority of in-vehicle systems \cite{nanjappan_towards_2019}. 
% While Tesla's design features a centrally placed screen for equal access by both driver and passenger, many other implementations place and angle screens toward the driver’s seat, reinforcing a driver-centric design paradigm \cite{nanjappan_towards_2019, detjen_how_2021}.

% Touchscreens, which have become the dominant form of interaction, are frequently angled toward the driver’s seat, reinforcing a driver-centric design paradigm \cite{nanjappan_touchscreen_2019, detjen_how_2021, dong_hmi_2019, khorsandi_tactile_2023}. For instance, the recently released Tesla Model 3 features a 15-inch touchscreen display that has replaced all traditional dashboard controls \cite{nanjappan_touchscreen_2019}. This trend is expected to continue, with other manufacturers increasingly adopting touchscreens for in-vehicle systems.

% +++++++++++++++++++++ V1 +++++++++++++++++++++
% While touchscreens allow for dynamic user interfaces \cite{detjen_how_2021}, they also introduce several challenges. From a driver’s perspective, touchscreen-based controls might increase the risk of distracted driving and fatigue \cite{khorsandi_tactile_2023}. Unlike physical buttons, which provide haptic feedback, touchscreens require constant visual attention, making it difficult for drivers to interact without looking away from the road \cite{detjen_how_2021, nanjappan_towards_2019}. This lack of tactility results in a higher cognitive workload, as users must rely on visual perception to confirm their actions \cite{detjen_how_2021, nanjappan_towards_2019}. Consequently, the absence of tactile cues in modern vehicles has raised concerns regarding usability and safety \cite{dong_disappearing_2019}.

% Beyond safety concerns, current in-vehicle interfaces are also not well-optimized for passengers. Most car interiors are designed with the assumption that the driver is the primary user, leaving passengers with limited interactive options \cite{khorsandi_interactive_nodate, inbar_make_2011}. Passenger-facing interfaces, such as rear-seat entertainment systems, are still rare and primarily found in premium vehicles \cite{berger_tactile_2019} (e.g. BMW i7 \cite{noauthor_2025_nodate}). As a result, many passengers resort to using their personal smartphones or tablets to stay engaged during travel \cite{inbar_make_2011, wilfinger_are_2011}. This however, could introduce additional problems, as looking down at a screen rather than in the direction of movement can induce motion sickness \cite{diels_self-driving_2016}. Furthermore, since most touchscreens are oriented toward the driver (e.g. Renault Megane E-Tech \cite{noauthor_all_nodate}), passengers often struggle to interact with built-in vehicle systems, limiting their ability to participate in collaborative tasks \cite{berger_tactile_2019} like inputting navigation directions \cite{inbar_make_2011}.

While touchscreens allow for dynamic user interfaces \cite{detjen_how_2021}, their reliance on visual attention and lack of tactile feedback raises significant usability and safety concerns for drivers \cite{detjen_how_2021, dong_disappearing_2019, nanjappan_towards_2019}. With the assumption that the driver is the primary user, vehicle manufacturers may try to mitigate these issues by positioning and orienting the touchscreens toward the driver, as seen in the Renault Megane E-Tech \cite{noauthor_all_nodate}, leaving other passengers with limited interactive options \cite{berger_tactile_2019, inbar_make_2011}.

While driver distraction due to touchscreen-based interfaces is no longer a concern in fully autonomous vehicles, the drawbacks of this in-vehicle interaction paradigm persist for passengers. Interacting with integrated touchscreens can be ergonomically challenging, requiring users to raise their arms, which becomes tedious over extended periods \cite{berger_tactile_2019}. Furthermore, direct-touch interactions in moving vehicles can result in reduced accuracy \cite{ng_evaluation_2017}, increased eye-glances toward the vehicle interior \cite{corsten_hapticase_2015} potentially increasing the risks of motion sickness \cite{diels_self-driving_2016}, and longer task completion times compared to traditional physical buttons \cite{ng_evaluation_2017}, ultimately diminishing the overall comfort of the ride.

Beyond these usability concerns, the look and feel of current in-vehicle interfaces remains fundamentally misaligned with the envisioned transformation of car interiors into multifunctional living spaces. While future vehicle cabins are expected to support relaxation and leisure activities, the materials commonly used in today’s interfaces, such as glass, metal, and plastic, are often perceived as impersonal, stiff, and cold \cite{brauner_interactive_2017}. This reliance on rigid controls can feel disconnected from the softer, more inviting materials typically associated with home and lounge environments \cite{schafer_whats_2023}. Moreover, existing in-car interfaces still prioritize functionality over user experience factors such as playfulness, engagement, and aesthetic appeal \cite{khorsandi_interactive_nodate}.

\sloppy
To accommodate these evolving requirements, future \gls{AV}s must integrate new interaction paradigms that move beyond traditional touchscreen-based controls. As vehicle interiors transform into multifunctional living spaces, there is a growing need for ubiquitous interfaces that are not only functional but also seamlessly embedded into the environment, providing intuitive, effortless, and hedonically rich interactions. In response to these challenges, this work explores a novel, unobtrusive, passenger-centric user interface for non-driving-related in-vehicle controls: an interactive non-wearable e-textile surface for user input, designed in conjunction with a windshield display. By leveraging the inherent tactility of textile materials, this approach aims to create a more natural and rich interaction experience, aligning with the vision of automated vehicles as adaptive and inviting spaces that support passengers to engage in a wide range of \gls{NDRA}s.

\subsection{Thesis Objective}

Although textiles are already familiar materials, omnipresent in our environment from clothing to home furniture \cite{schafer_whats_2023}, interactive textiles as user interfaces remain a relatively unexplored concept in consumer products. As a result, most users have likely never encountered e-textiles in an interactive capacity, meaning they have not yet gathered knowledge of how to interact with them.

In the \gls{HMI} research community, there is a shared belief that \textit{“such interfaces must not rely on users getting training or explicit instructions or manuals to successfully interact with them”} \cite[pp.~1159]{mlakar_exploring_2021}. However, this raises a critical challenge: How can novice, untrained users intuitively understand and interact with textile interfaces? While research on enhancing the intuitiveness of textile interfaces, either by leveraging textile affordances \cite{mlakar_exploring_2021} or integrating tactile cues \cite{nowak_shaping_2022} is emerging, existing design guidelines remain limited \cite{schafer_whats_2023} and often focus on isolated aspects rather than providing a holistic approach. 
Additional feedforward cues show potential in communicating the interactive affordances of textile interfaces to users \cite{dong_disappearing_2019}, yet their role has not been investigated in a comprehensive or systematically differentiated manner.

Furthermore, research on textile interfaces in the context of vehicles is still scarce. The only existing study \cite{khorsandi_fabricar_2023} primarily addresses driver distraction reduction and limits each textile surface to a single interaction or function, which restricts the flexible application of the interface and limits its potential to support complex interactions within in-vehicle \gls{NDRA}s. Moreover, these investigated implementations would still require explicit introduction and guidance by the study conductors (e.g., verbal explanation or demonstration) as the interfaces were not designed to be discoverable for first-time users. In addition, hedonic aspects of user experience, such as playfulness or emotional engagement remain largely unaddressed.

%Moreover, many of these implementations still require onboarding (explicit introductions by the study conductors), making them less intuitive for first-time users. 
%To the best of our knowledge, our work is the first to propose a holistic textile interface for AVs, offering a range of functionalities through a diverse set of interactions, ultimately aiming for a more seamless and intuitive user experience.

This thesis aims to address these gaps by exploring how non-wearable textile interfaces can be designed to support intuitive and emotionally engaging interactions in the context of \gls{AV}s. By investigating the role of different levels of feedforward, this work seeks to understand how such interfaces can guide first-time users without the need for prior instruction or training.

%This is especially relevant in the context of shared autonomous vehicles (SAVs), where users may frequently encounter unfamiliar interfaces in vehicles they do not own. In such scenarios, intuitiveness becomes crucial, as there is no opportunity for onboarding, customization, or prolonged familiarization.

% +++++++++ OLD GOAL +++++++++
% The central objective is to identify how feedforward cues, ranging from inherently perceivable tactile and visual properties to augmented seamlessly embedded abstract or external, explicit semantic visual signals, can be leveraged to improve discoverability, support intuitive gesture mappings, and enrich the overall user experience. The goal is to find a balance: providing just enough guidance to make interaction feel effortless and self-evident, while preserving the novelty and emotional appeal that makes the interaction feel engaging, natural and “magical.”
The central objective is to identify how feedforward cues, ranging from inherently perceivable tactile properties to augmented visual signals, can be leveraged to improve discoverability, support intuitive use, and enrich the overall user experience. The goal is to determine which level of guidance is most effective at making the interaction feel effortless and self-evident, while simultaneously enhancing the novelty and emotional appeal that makes the experience engaging and memorable

In doing so, this thesis contributes to the broader field of automotive \gls{HMI} by offering a more nuanced understanding of how interactive textiles can move beyond single-function controls and instead become flexible, expressive interfaces suited for the multifunctional and adaptive interior spaces of future \gls{AV}s.

\begin{comment}
To the best of our knowledge, our work is the first to propose a holistic textile interface for NDRTs in AVs that not only supports a wide range of functionalities through diverse interactions but also eliminates the need for onboarding. By exclusively leveraging inherent textile affordances and built-in feedforward cues, our approach enables users to intuitively understand and interact with the interface from the very first use, ensuring a seamless and natural experience.

Building on these considerations, our work aims to explore how non-wearable textile interfaces can serve as intuitive and unobtrusive interaction surfaces for non-driving-related tasks (NDRTs) in automated vehicles. Specifically, we investigate which interaction metaphors best support natural and intuitive gestures (RQ1) and how inherent and augmented feedforward cues can enhance the discoverability and usability of these interactions (RQ2). Furthermore, we examine whether a gesture-based textile interface, in conjunction with a windshield display (WSD), can enable unambiguous, intuitive interaction without requiring prior onboarding or explicit instructions (RQ3). Finally, we seek to understand how user interaction with such an interface evolves over time, particularly in terms of reduced reliance on visual feedback and increased dependence on muscle memory and haptic perception (RQ4). By addressing these research questions, this thesis aims to contribute to the development of seamless, passenger-centric interaction paradigms that align with the vision of future automated vehicle interiors as adaptive and engaging environments.
\end{comment}

\begin{comment}    
Given these limitations, there is a growing need to rethink vehicle interior design to accommodate the needs of all occupants, not just the driver. As automation advances and vehicles transition into shared, multi-purpose spaces, designing interfaces that support both driver and passenger interactions will be essential. Addressing these challenges requires exploring alternative interaction paradigms that go beyond touchscreens, leveraging haptic and textile-based interfaces to create more intuitive and inclusive user experiences.

Although driver distraction due to visual demand or cognitive workload is irrelevant in fully autonomous vehicles, the disadvantages of current car interior design persist for passengers. The use of touchscreens, whether integrated into the vehicle or as mobile devices, creates several ergonomic issues. Interacting with built-in touchscreens can be tedious due to the need to raise the arm, while mobile device usage can lead to motion sickness as users look down instead of in the direction of movement \cite{berger_tactile_2019}. Furthermore, direct touch interfaces reduce accuracy, increase the need for visual attention, and prolong task completion times compared to physical buttons \cite{berger_tactile_2019}.

Additionally, future vehicle interiors are expected to evolve into living spaces, emphasizing comfort and leisure. However, current in-vehicle interfaces remain rigid and impersonal. Modern smart home gadgets, typically made of glass, metal, and plastic, are often perceived as cold and digital \cite{brauner_smart_2017}. Similarly, traditional in-car controls, such as rigid plastic buttons, do not always align with the aesthetic and tactile expectations of users \cite{schaefer_tactile_2023}. Current user interfaces tend to overlook values like playfulness, engagement, and aesthetics, which are crucial for creating an enjoyable and immersive passenger experience \cite{khorsandi_nabil_2023}.
\end{comment}

\begin{comment}
    
% "With the advent of automated driving the requirements considerably shift for automotive user interfaces (UIs)". \cite{pfleging_investigating_2016}
%"Automation will allow entirely new use of time and space in the car and significantly impact system design". \cite{detjen_how_2021}
The advent of automated driving will allow entirely new use of time and space in the car \cite{detjen_how_2021}, considerably shifting/impacting the requirements for automotive user interfaces (UIs) \cite{pfleging_investigating_2016}. 
% This allows drivers to draw their full attention to other tasks than driving.” \cite{pfleging_investigating_2016}
% Therefore it can be expected, that "In the near future cars will transform into living spaces rather than just means of transportation". \cite{schartmuller_automated_2020}
% And “modifications on the design of vehicle interiors are opening the door for unprecedented non-driving activities and transforming cars from transportation vehicles to living spaces [107].” \cite{khorsandi_fabricar_2023}
Automated vehicles (AVs) will allow drivers to shift their attention from the driving to non-driving related tasks/activities \cite{pfleging_investigating_2016} \cite{kun_shifting_2016} which could transform cars from transportation vehicles to living spaces in the near future \cite{schartmuller_automated_2020} \cite{khorsandi_fabricar_2023} completely reshaping vehicle interiors and how users interact with them.

% The findings indicate that besides traditional activities (talking  to passengers, listening to music), daydreaming, writing text  messages, eating and drinking, browsing the Internet, and  calling are most wanted for highly automated driving. This  shows the potential for mobile and ubiquitous multimedia  applications in the car. \cite{pfleging_investigating_2016}
% Leisure activities (88%), resting (75%), socialising (70%) and productive activities (55%) were the most popular NDRTs amongst survey participants who were more likely to own an AV  \cite{wilson_non-driving_2022}
Activities passengers of AVs will likely adopt in the future are leisure activities, including relaxing, daydreaming, sleeping, listening to audible content, talking to passengers, eating and drinking, looking out of the window or the use of mobile devices \cite{pfleging_investigating_2016}. Ubiquitous user interfaces show potential to support these kind of activities in vehicle interiors, by being more in the background and not center of those activities.

%In particular "textiles reveal new ways of interaction and may support future innovative applications in the area of ubiquitous computing" ([7]\cite{mlakar_design_2020})
% “Utilizing fabric and leather surfaces as seamless car interfaces can be one of the ways to engage users in ubiquitous non-driving-related activities (NDRAs) within future car interiors.” \cite{khorsandi_fabricar_2023}
Especially interfaces using smart textiles have the potential to be one of the more seamless [44] ways of interacting with new technology [28, 35], as textiles are an already familiar and comfortable context. \cite{mlakar_exploring_2021} since car interior are currently already designed with a myriad of fabric surfaces \cite{khorsandi_fabricar_2023} (which can also be found at home, so that ads to the living space feel). 
While textiles show potential as an unobtrusive input modality, a visual output modality could be HUD. 
\end{comment}

% \subsection{Outline}
% The thesis is structured as follows:
